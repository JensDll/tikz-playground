\documentclass[
  ngerman,
  a4paper,
  12pt
]{article}

\usepackage{tikz}
\usepackage{xcolor}
\usepackage[utf8]{inputenc}
\usepackage[english, german]{babel}
\usepackage{amsmath}
\usepackage{etoolbox}
\usepackage{xifthen}

\usetikzlibrary{
  positioning,
  calc,
  shapes.multipart,
  external,
  arrows.meta,
  fit,
  graphs
}

\tikzset{
  >=Triangle,
  n/.style = {
    circle, draw=black!30, line width=1pt
  },
  w/.style = {
    n, fill=green!10
  },
  b/.style = {
    n, fill=yellow!20
  },
  pics/nnn/.style = {
    code = {
      \node [circle split, draw=black!30, line width=1pt, fill=blue!10] (-n) {
        $a$
        \nodepart{lower}
        $\sum$
      };
    }
  }
}

\tikzexternalize[prefix=figures/]

\begin{document}

\tikzsetnextfilename{nn-neuron}
\begin{center}
\begin{tikzpicture}
  \path
    (0,0) node [] (x1) {$x_1$}
    (1.2,0) node [] (x2) {$x_2$}
    (2.2,0) node [] {\ldots}
    (3.2,0) node [] (xn) {$x_n$}

    (0,1.2) node [w] (w1) {$w_1$}
      edge [<-] (x1)
    (1.2,1.2) node [w] (w2) {$w_2$}
      edge [<-] (x2)
    (2.2,1.2) node [] {\ldots}
    (3.2,1.2) node [w] (wn) {$w_n$}
      edge [<-] (xn)

    (1.6,3) pic [] (nnn) {nnn}
      (nnn-n.250) edge [<-] (w1)
      (nnn-n.270) edge [<-] (w2)
      (nnn-n.290) edge [<-] (wn)
    (1.6,4.5) node [] {$y$}
      edge [<-] (nnn-n.north)
    
    (4.2,0)   node [anchor=west] {Eingaben}
    (4.2,1.2) node [anchor=west] {Gewichte}
    (4.2,2.95) node [anchor=west] {
      $\begin{aligned}
        &\text{Aktivierungsfunktion:} &y &= a(z) \\
        &\text{Gewichtete Summe:} &z &= \mathbf{x}^T\mathbf{w}
      \end{aligned}$
    }
    (4.2,4.5) node [anchor=west] {Ausgabe};
\end{tikzpicture}
\end{center}

\hrule

\tikzsetnextfilename{example-nn}
\begin{center}
\begin{tikzpicture}[
  every node/.style={
    minimum size=8mm
  }
]
  \def\gap{2};

  % Input layer
  \foreach \i / \pos in {
    {0/(0,0)},
    {1/(1.5,0)},
    {2/(3,0)}}
  {
    \ifthenelse{0 = \i}{
      \node [b] (i\i) at \pos {1};
    }{
      \node [n] (i\i) at \pos {$x_\i$};
    }
  }

  % Hidden 1
  \foreach \i / \pos in {
    {0/(0,\gap)},
    {1/(1.5,\gap)},
    {2/(3,\gap)},
    {3/(4.5,\gap)},
    {4/(6,\gap)}}
  {
    \ifthenelse{0 = \i}{
      \node [b, xshift=(4 - 2) * -0.75cm] (h1\i) at \pos {1};
    }{
      \node [n, xshift=(4 - 2) * -0.75cm] (h1\i) at \pos {};
    }
  }

  % Hidden 2
  \foreach \i / \pos in {
    {0/(0,2 * \gap)},
    {1/(1.5,2 * \gap)},
    {2/(3,2 * \gap)},
    {3/(4.5,2 * \gap)}}
  {
    \ifthenelse{0 = \i}{
      \node [b, xshift=(4 - 3) * -0.75cm] (h2\i) at \pos {1};
    }{
      \node [n, xshift=(4 - 3) * -0.75cm] (h2\i) at \pos {};
    }
  }

  % Output Layer
  \node [n, xshift=3 * 0.75cm] (o0) at (0, 3 * \gap) {$y_1$};

  % Connect Input and Hidden 1
  \foreach \i in {0,...,2}
  {
    \foreach \j in {1,...,4}
    {
        \draw [->] (i\i) -- (h1\j);   
    }
  }

  % Connect Hidden 1 and Hidden 2
  \foreach \i in {0,...,4}
  {
    \foreach \j in {1,...,3}
    {
        \draw [->] (h1\i) -- (h2\j);
    }
  }

  % Connect Hidden 2 and Output
  \foreach \i in {0,...,3}
  {
    \foreach \j in {0,...,0}
    {
        \draw [->] (h2\i) -- (o\j);   
    }
  }

  % Labels
  \path
    (5.5,0) node [anchor=west] {Eingabeschicht}
    (5.5,\gap) node [anchor=west] {Verdeckte Schicht 1}
    (5.5,2 * \gap) node [anchor=west] {Verdeckte Schicht 2}
    (5.5,3 * \gap) node [anchor=west] {Ausgabeschicht};
\end{tikzpicture}
\end{center}

\end{document}